\chapter{Анализ предметной области}

В настоящее время выделяют 4 основных подхода к определению тональности текста~\cite{big}.

Математический подход --- создание музыкального фрагмента на основе математических алгоритмов~\cite{chew_mathematical_computational_modeling}.

Статистический подход --- создания новых музыкальных фрагментов на основе анализа статистических свойств музыкальных данных таких как жанр, стиль, стиль композитора~\cite{conklin_music_generation}.

Грамматический подход --- создание музыкального фрагмента на основе грамматических правил и шаблонов для создания музыкальных структур, таких как мелодии, аккорды или ритмы~\cite{grammer}.

Трансляционный подход --- это способ генерации музыкальных произведений на основе немузыкальных данных, таких как графические образы или текст.
Данный процесс может быть случайным или основываться на определенных правилах мелодической структуры, с использованием нейросетевых технологий для преобразования и распознавания исходных данных~\cite{translation}.

\chapter{Описание существующих решений}

%Андрей
\section{Математический подход}

Задача генерации музыкальных фрагментов стала актуальной, начиная со второй половины 20 века. Под музыкальным фрагментом подразумевают последовательность нот и молчания, где нотой является музыкальный звук, а молчанием отсутствие звуков. Музыкальность или гармоничность фрагмента определяется субъективно с помощью социального опроса. Одними из популярных математических методов, использующих математические алгоритмы при генерации музыкальных последовательностей, являются метод Бьорклунда и метод, использующий клеточные автоматы.

\subsection{Алгоритм Бьорклунда}
Музыкальный фрагмент из одного музыкального звука представляет из себя циклическую последовательность нот и молчания, расположенных "равномерно". Для того, чтобы сформировать последовательность используется алгоритм Бьорклунда. Введем два обозначения <<X>> и <<•>> где первое обозначает начало ноты, а второе представляет молчание. Тогда музыкальная последовательность длиной N временных интервалов может быть представлена в виде последовательности из N символов. Обозначим количество X за K, а длину последовательности за N. Примерами таких последовательностей являются кубинские ритмы Тресильо (N = 8, K = 3) и Чинквильо (N = 8, K = 5)~\cite{bjorklund}, пояснение изображено на рисунке~\ref{img:rhytms}.

\includeimage
{rhytms}
{f}
{H}
{0.7\textwidth}
{Тресильо (a) и Чинквильо (b)}

Описание алгоритма генерации музыкального фрагмента с помощью алгоритма Бьорклунда:
\begin{enumerate}
	\item определить три группы: группу А, состоящую из символов <<Х>>, группу В, состоящую из символов <<•>> и пустую группу С;
	\item очистить содержимое группы С;
	\item получить новый элемент, сопоставив элементу группы А  в соответствие элемент группы В,  новый элемент поместить в группу С;
	\item повторить шаг 3, если группы А и В не пустые;
	\item содержимое группы С записать в группу А, а оставшиеся элементы из групп А или В, записать в группу В;
	\item перейти на шаг 2, если группа В не пустая;
	\item соединить все элементы из группы А в один. Сформировать музыкальный фрагмент на основе последовательности символов <<X>> и <<•>> в получившемся элементе, размещая вместо <<X>> ноту, а вместо <<•>> молчание.
\end{enumerate}
В случае, если N делится на K без остатка, решение тривиально. Пример для N = 6, K = 2 приведен в таблице~\ref{tbl:ex1}.
\begin{table}[H]
	\caption{Пример для N = 6, K = 2}
	\label{tbl:ex1}
	\begin{tabular}{|c|c|c|c|c|}
		\hline
		\makecell{Номер шага \\ выполнения} & Номер шага в алгоритме  & A & B &   C\\
		\hline
		1 & 1 & X, X & •, •, •, • &  \\
		\hline
		2 & 2 & X, X & •, •, •, • &  \\
		\hline
		3 & 3 & X & •, •, • & X• \\
		\hline
		4 & 4 & X & •, •, • & X• \\
		\hline
		5 & 3 &  & •, • & X•, X• \\
		\hline
		6 & 4 &  & •, • & X•, X•\\	
		\hline
		7 & 5 & X•, X• & •, • & X•, X• \\
		\hline
		8 & 6 & X•, X• & •, • & X•, X• \\
		\hline
		9 & 2 & X•, X• & •, • &  \\
		\hline	
		10 & 3 & X• & • & X•• \\
		\hline
		11 & 4 & X• & • & X•• \\
		\hline
		12 & 3 &  &  & X••, X•• \\
		\hline
		13 & 4 &  &  & X••, X•• \\
		\hline
		14 & 5 & X••, X•• &  & X••, X•• \\
		\hline
		15 & 6 & X••, X•• &  & X••, X•• \\
		\hline
		16 & 7 & X••X•• &  & X••, X•• \\
		\hline
	\end{tabular}
\end{table}
Пример для N = 5, K = 2 приведен в таблице~\ref{tbl:ex2}.
\begin{table}[H]
	\caption{Пример для N = 6, K = 2}
	\label{tbl:ex2}
	\begin{tabular}{|c|c|c|c|c|}
		\hline
		\makecell{Номер шага \\ выполнения} & Номер шага в алгоритме & A & B & C \\
		\hline
		1 & 1 & X, X & •, •, • &  \\
		\hline
		2 & 2 & X, X & •, •, • &  \\
		\hline
		3 & 3 & X & •, • & X• \\
		\hline
		4 & 4 & X & •, • & X• \\
		\hline
		5 & 3 &  & • & X•, X• \\
		\hline
		6 & 4 &  & • & X•, X• \\
		\hline
		7 & 5 & X•, X• & • & X•, X• \\
		\hline
		8 & 6 & X•, X• & • & X•, X• \\
		\hline
		9 & 2 & X•, X• & • &  \\
		\hline
		10 & 3 & X• &  & X•• \\
		\hline
		11 & 4 & X• &  & X•• \\
		\hline
		12 & 5 & X•• & X• & X•• \\
		\hline
		13 & 6 & X•• & X• & X•• \\
		\hline
		14 & 2 & X•• & X• &  \\
		\hline
		15 & 3 &  &  & X••X• \\
		\hline
		16 & 4 &  &  & X••X• \\
		\hline
		17 & 5 & X••X• &  & X••X• \\
		\hline
		18 & 6 & X••X• &  & X••X• \\
		\hline
		19 & 7 & X••X• &  & X••X• \\
		\hline
	\end{tabular}
\end{table}

\subsection{Алгоритм, основанный на клеточных автоматах}
Более сложные музыкальные фрагменты можно генерировать с помощью клеточных автоматов. Клеточные автоматы -- это модели вычислений, которые состоят из сети ячеек (клеток)~\cite{man}, каждая из которых может находиться в определенном состоянии, и эти состояния изменяются в соответствии с правилами, определенными для данного автомата. Пример одномерного клеточного автомата приведен на рисунке~\ref{img:ca}. 
\includeimage
{ca}
{f}
{H}
{0.7\textwidth}
{Пример одномерного клеточного автомата}

В контексте генерации музыки, каждая ячейка клеточного автомата может рассматриваться как нота или аккорд. 
Описание алгоритма генерации музыкального фрагмента с помощью клеточного автомата:
\begin{enumerate}
	\item определить размер и мерность сетки (1D или 2D) клеточного автомата;
	\item задать правила, которые определяют, какое состояние будет принимать каждая клетка в следующем поколении, исходя из ее текущего состояния и состояния соседних клеток;
	\item каждой клетке сопоставить определенную ноту;
	\item задать условие остановки вычислений;
	\item выполнить вычисления в клеточном автомате, чтобы определить состояние каждой клетки в следующем поколении, при каждом переходе в следующее поколение записать в результат музыкальное событие на основе состояния клетки;
	\item продолжить вычисления, пока ложно условие остановки.
\end{enumerate}

%ПАША
\section{Статистический подход}

Музыкальное произведение может быть представлено последовательностью событий, в котором ноты являются музыкальными объектами с длительностью и временем начала~\cite{statistic}. Статистическая модель музыки приписывает каждому возможному музыкальному событию вероятность его появления. Наиболее распространенным типом статистических моделей, встречающихся для музыки, как для анализа, так и для синтеза, являются модели, которые присваивают вероятности событиям, обусловленным только более ранними событиями в последовательности~\cite{statistic}. Создание фрагмента из статистической модели означает выборку фрагмента, который имеет высокую вероятность в соответствии с моделью~\cite{statistic}. Существуют разные методы генерации музыкальных композиций, основывающиеся на статистическую модель. К таким методам относятся метод случайного блуждания и Марковский метод.


\subsection{Метод случайного блуждания}
Это способ создания музыки из модели, состоящий в выборке на каждом этапе случайного события из распределения событий на этом этапе~\cite{statistic}. После того, как событие обработано, оно добавляется к фрагменту, и процесс продолжается до тех пор, пока не будет превышена заданная длительность фрагмента~\cite{rwalk}. Рассмотрим схематичное представление модели случайного блуждания, представленную на рисунке~\ref{img:rwalk}


\includeimage
{rwalk}
{f}
{H}
{1\textwidth}
{Схематичное представление метода случайного блуждания}


${x1, x2, …, xn}$ – множество событий, после обработки путем перехода между событиями получаем музыкальный фрагмент.

Пусть точечная частица может совершать только один тип движений. В дискретные моменты времени ${t0, t1, ..., tn} $ частица совершает скачок вдоль прямой так, 
то в момент времени $tn+1 $ она оказывается в точке, отстоящей на единичное расстояние вверх или вниз от точки, где она находилась в момент времени.
На рисунке~\ref{img:randwalk} визуализирована работа метода случайного блуждания.

\includeimage
{randwalk}
{f}
{H}
{1\textwidth}
{Визуализация работы метода случайного блуждания}


Без ограничения общности можно считать, что координата частицы в любой момент времени есть целое число.
Введём на прямой некоторое начало отсчёта и будем писать $\xi  * j = m$, если в момент времени $tj$ частица находилась в точке $m$; здесь $j = 1, 2, ..., n$  и  $m = x1, x2, ..., xn$.

Предположим, что блуждание имеет случайный характер:
прыжок вверх частица совершает с вероятностью $p$, а прыжок вниз — с вероятностью $q$.
При этом любые другие перемещения невозможны~\cite{rwalk}, так что $p + q = 1$.
Примем также, что вероятности скачков не зависят от положения частицы и предыстории её движения.


При анализе случайных блужданий частицы очень удобно пользоваться понятием траектории её движения за $n$ шагов~\cite{rwalk}.
Она представляет собой набор точек $(j, \xi * j)$, $j = 1,2,...,n$ на двумерной координатной плоскости,
в котором первая координата — это номер члена последовательности, т. е. по сути момент времени $t = j$, а вторая -- величина, значение которое равно координате частицы в момент времени $t = j$.
Для наглядности удобно соединить точки траектории отрезками прямых, на графике получится непрерывная ломаная из n звеньев,
координаты узлов которой $(j, \xi * j)$, $j = 0, 1, . . . , n$.
Последовательность этих узлов и будет являться искомым фрагментом~\cite{rwalk}.

Метoд случaйного блуждaния, хотя и примeним к музыкaльным импровизационным систeмам в реальном времени,
требующим быстрой и немедленной реакции системы имеет недостаток в создании законченных кусков, потому что он не может гарантировать,
что будут произведены куски с высокой общей вероятностью~\cite{statistic}. 

\subsection{Цепи Маркова}
Цепь Маркoва -- это модeль, описывающая последовательность возможных событий~\cite{mchains}. 
Эта последовательность должна удовлетворять предположению Маркова -- вероятность следующего состояния зависит от предыдущего состояния, а не от всех предыдущих состояний в последовательности~\cite{mchains}. 
Таким образом, композиция разбивается на последовательность состояний, и каждое состояние имеет набор возможных следующих состояний, которые выбираются на основе заданных вероятностей перехода. 
На рисунке~\ref{img:mc} изображено схематичное представление модели Марковских цепей.

\includeimage
{mc}
{f}
{H}
{1\textwidth}
{Схематичное представление метода Марковских цепей}


${x1, x2, …, xn}$ --- множeство событий. События могут быть определены различными способами в зависимости от потребностей модели~\cite{mchains}. 
Например, состоянием может быть нота, аккорд, длительность, интонация и так далее.  С каждым состоянием могут быть связаны вероятности перехода к следующим состояниям.
Эти вероятности могут быть выведены из обучающего набора данных, на основе которых создается модель.
Обучение модели может быть выполнено путем анализа существующих музыкальных композиций, при помощи которых определяются вероятности переходов.

Алгоритм Мaрковских цепей может быть записан в следующем виде~\cite{markov}:
\begin{itemize}
	\item подать на вход множество событий и размер необходимого фрагмента;
	\item рассчитать распределение вероятностей для событий;
	\item определить первое событие или сделать случайный выбор;
	\item сделать случайный выбор следующего события с учетом распределения вероятностей;
	\item повторять два предыдущих шага для сгенерированного события пока не достигнут размер нужный размер.
\end{itemize}

Разберем работу алгоритма на примере. В качестве входного множества событий возьмём корпус аккордов группы Beatles в композиции Yesterday~\cite{yestrday}:
[<<F>>, <<Em7>>, <<A7>>, <<Dm>>, <<Dm7>>, <<Bb>>, <<C7>>, <<F>>, <<C>>, <<Dm7>>,...]

Делаем биграммы (последовательность из двух соседних элементов) из соседних аккордов
[<<F Em7>>, <<Em7 A7>>, <<A7 Dm>>, <<Dm Dm7>>, <<Dm7 Bb>>, <<Bb C7>>, ...]

В качестве начального аккорда последовательности выбираем аккорд F. Необходимо рассчитать вероятности следующего аккорда. Есть 18 биграмм, которые начинаются с аккорда F.
[<<F Em7>>, <<F C>>, <<F F>>, <<F Em7>>, <<F C>>, <<F A7sus4>>, <<F A7sus4>>, ...]

Необходимо рассчитать частоту появления каждого уникального биграмма в последовательности.
{<<F Em7>>: 4, <<F C>>: 4, <<F F>>: 3, <<F A7sus4>>: 4, <<F Fsus4>>: 2, <<F G7>>: 1}

Если нормализовать полученное количество появлений биграммов в последовательности, получим вероятности

{<<F Em7>>: 0.222,  <<F C>>: 0.222,  <<F F>>: 0.167,  <<F A7sus4>>: 0.222,  <<F Fsus4>>: 0.111, <<F G7>>: 0.056}

Данные вероятности можно интерпретировать в виде графа, представленного на рисунке~\ref{img:chainsgraph}:

\includeimage
{chainsgraph}
{f}
{H}
{1\textwidth}
{Граф вероятностей перехода}

Каждый узел этого графа, кроме начального узла F в центре, представляет возможные состояния, которых может достичь наша последовательность, в нашем случае это аккорды, которые могут следовать за F. 
Некоторые из аккордов имеют более высокую вероятность, чем другие, некоторые аккорды не могут следовать за F. Например, аккорд Am, потому что не было биграмма, которой объединяет этот аккорд с F. 
Далее нужно выполнить случайную выборку в соответствии с распределением вероятности. Допустим, наш результат этого случайного выбора был Em7. 
Теперь у нас новое состояние, и мы можем повторить весь процесс снова.

Таким образом, метод генерирует музыку, которая статистически "похожа" на исходные композиции, но в то же время имеет свою уникальность, так как варианты переходов могут быть случайными.


%СТЕПА 
\section{Грамматический подход}

Грамматическая модель может определять, какие ноты, аккорды или мелодические фразы могут быть использованы в композиции, а также как они могут быть комбинированы и повторены. Она может также определять структуру композиции, такую как введение, куплеты, припевы и заключение. Применение грамматической модели в музыке позволяет создавать композиции, которые соответствуют определенным стилям и жанрам. Грамматическая модель применяется для автоматической генерации музыки, так и как средство для анализа и понимания уже существующих музыкальных произведений. Существуют разные методы генерации музыкальных композиций, основывающиеся на грамматической модели. К таким методом относится порождающая грамматика~\cite{haskell}.

\subsection{Порождающая грамматика}
Этот метод является формализмом генеративной лингвистики, связанный с изучением синтаксиса. В рамках подхода порождающей грамматики формулируется система правил, при помощи которых можно определить, какая комбинация слов оформляет грамматически правильное предложение. Термин впервые введён в научный оборот американским лингвистом Ноамом Хомским в конце 1950–х годов. Цель лингвистической теории по Хомскому заключается в том, чтобы объяснить факт поразительно быстрого усвоения родного языка ребенком на основе явно недостаточного внешнего стимула, то есть той информации, которая может быть извлечена из речи окружающих~\cite{methods}. Данный метод построения грамматики называется методом динамически расширяющегося контекста. В контексте музыки, порождающая грамматика может определять правила для порождения последовательности нот или аккордов. Например, она может иметь правило, которое генерирует последовательность нот в рамках заданного тонального ряда или музыкальной шкалы. Она также может иметь правила для определения длительности нот или мелодических фраз.

Рассмотрим пример составления грамматических правил и генерации новой строки по этим правилам. Берется обычная строка: ABCDEFGIKFHLEFJ. И строится для нее грамматика, начав, например, с символа F~\cite{notes}. Записывается правило, которое бы указывало, какую букву следует поставить, если встречен символ F. Нельзя создать такое правило, так как лишь по одной букве нельзя определить, что должно идти следом: после F может идти как G, так и H или J. Поэтому добавляется контекст к букве F, контекст — это символы, окружающие F. Взяв по одной букве перед F, получается EF и KF. Контекстом для буквы F служат буквы E и K. Таким образом, расширяется контекст на один символ, поэтому данный метод построения грамматики называется методом динамически расширяющегося контекста. Это правила для буквы F, в зависимости от ее контекста выбирается какое-то одно правило. Процесс генерации новой строки выглядит следующим образом: дана начальная последовательность, например, ADEF. Буквы берутся с конца. F — нет правила с такой левой частью, расширяется контекст — EF, опять нет, расширяется — DEF, есть такое правило, ставится G, получается ADEFG. Начинается все сначала: берется буква G и т. д. столько раз, сколько нужно.

Порождающая грамматика используется для задачи генерации музыкальной партитуры в виде нот, однако требует определения четких формальных правил построения композиции, что является крайне трудоемким процессом.

\subsection{L-системы}
L-система или система Линденмайера — это параллельная система переписывания и вид формальной грамматики. L-системой (точнее, её разновидностью, детерминированной контекстно независимой L-системой) называют набор, состоящий из алфавита, аксиомы, и множества правил. Алфавитом называется конечное множество, а его элементы — символами. Природа символов не важна, их единственная функция — отличаться друг от друга. Строкой над алфавитом является конечная последовательность символов алфавита. Аксиома — это некоторая строка над алфавитом. Каждое правило — это пара, состоящая из предшественника и последователя. Предшественник — это символ алфавита, а последователь — строка над алфавитом. Пример пары:

C → EDEC + FEC + ED - EC,

стрелка отделяет предшественника от последователя. В списке правил символы-предшественники должны быть уникальными.

\begin{table}[H]
	\centering
	\caption{Правила L-системы}
	\begin{tabular}{|c|c|c|}
		\hline
		\makecell{Алфавит} & {Аксиома}  & {Правила}\\
		\hline
		{C,D,E,F,G,A,B} & ED & \makecell{C → EDEC + FEC + ED - EC, \\ D → ED + EC - ED - GEDEC, \\ E →"" (последователь -- пустая строка), \\ F → B, \\ G → A} \\
		\hline
	\end{tabular}
\end{table}

Как только L-система определена, она начинает развиваться в соответствии с её правилами. Начальным состоянием L-системы является её аксиома. При дальнейшем развитии эта строка, описывающая состояние, будет меняться. Развитие L-системы происходит циклически. В каждом цикле развития строка просматривается от начала к концу, символ за символом. Для каждого символа ищется правило, для которого этот символ служит предшественником. Если такого правила не нашлось, символ оставляется без изменений. Иными словами, для тех символов  X, для которых нет явного правила, действует неявное:  X→X. Если же соответствующее правило найдено, символ-предшественник заменяется на строку-последователь из этого правила.

Для иллюстрации рассмотрим следующую L-систему (она называется \textbf{Algæ} — водоросль, поскольку её развитие моделирует рост одного из видов водорослей):

\begin{table}[H]
		\centering
	\caption{L-система водоросль}
	\begin{tabular}{|c|c|}
		\hline
		Аксиома & Правила \\
		\hline
		A & \makecell{A → B \\ B → AB} \\
		\hline
	\end{tabular}
\end{table}

Состояния этой L-системы, соответствующие первым десяти циклам развития системы:
\begin{table}[H]
	\centering
	\caption{Состояния L-системы}
	\begin{tabular}{|c|p{12cm}|}
		\hline
		Поколение & Состояние \\
		\hline
		0 & A \\
		\hline
		1 & B \\
		\hline
		2 & AB \\
		\hline
		3 & BAB \\
		\hline
		4 & ABBAB \\
		\hline
		5 & BABABBAB \\
		\hline
		6 & ABBABBABABBAB \\
		\hline
		7 & ABABBABABBABBABABBAB \\
		\hline
		8 & ABBABBABABBABBABABBABABBABBABABBAB \\
		\hline
		9 & BABABBABABBABBABABBABABBABBABABBAB
		BABABBABABBABBABABBAB \\
		\hline
	\end{tabular}
\end{table}

Длины строк, кодирующих состояние такой L-системы, образуют последовательность чисел Фибоначчи, то есть такую числовую последовательность, в которой каждое число равняется сумме двух предыдущих. Последовательностями Фибоначчи будут также количества символов A и B в этих строках. В последовательности строк имеется та же закономерность, что и в последовательности чисел Фибоначчи: каждая строка является «суммой» (конкатенацией) двух предыдущих~\cite{l-systems}.

% мое 
\section{Трансляционный подход}
Создание музыки творческий процесс, его автоматизация сложна из-за важной роли композитора и труднопонимаемой эмоциональности в музыке~\cite{big}.
Для автоматической генерации музыкальных композиций с учетом эмоционального состояния пользователя-композитора можно использовать трансляционные модели~\cite{web}.

Генерация музыки на основе изображения может применяется в компьютерных играх, рекламе и фильмах для создания фоновой музыки. 
Автоматизация этого процесса позволит сократить расходы компаний, учитывая небольшие требования к фоновой музыке в этих областях~\cite{actuality}.

%todo
%Целью данной работы является описание метода генерации фрагмента музыкального произведения по изображению.

Формальная постановка задачи: 
\begin{enumerate}
	\item пусть $A$ --- множество всех возможных изображений, исходные данные для алгоритма;
	\item пусть $B$ --- множество всех возможных музыкальных композиций, результаты работы алгоритма;
	\item Тогда задача сводится к поиску приближенной функции, которая каждому элементу (изображению) $a \in A$ сопоставляет множество музыкальных композиций $f(a) \subseteq B$. Формула~(\ref{eq:test}) формализует данное многозначное отображение.
	\begin{equation}
		f: A \rightarrow 2^B 
		\label{eq:test}
	\end{equation}
	%многозначное отображение
\end{enumerate}


%todo
%В данном разделе описан метод генерации фрагмента музыкального произведения по изображению, алгоритм анализа изображения, 

Генерация фрагмента музыкального произведения по изображению, представляет собой преобразование визуальных данных в последовательности нот с определенным тоном и темпом~\cite{alg}. 

Метод генерации фрагмента музыкального произведения по изображению состоит из двух составляющих алгоритмов: 
\begin{enumerate}
	\item алгоритм анализа изображения; % спросить про алгоритм
%todo
	\item алгоритм генерации музыкального фрагмента.
\end{enumerate}	

\subsection{Алгоритм анализа изображения}

%todo
Алгоритм анализа изображения состоит из следующих шагов~\cite{alg}.
\begin{enumerate}
	\item Преобразовать входное изображение из цветовой модели RGB в HSB.
	Цветовая модель HSB более удобна, так как содержит необходимые характеристики.
	\item Определить тональность произведения.
	\item Определить жанр произведения.
	\item Определить схему соотнесения цвета и ноты.
	\item Используя выбранную схему, найти последовательность первых нот, считанных с
	изображения.
\end{enumerate}

Алгоритм анализа изображения позволяет извлекать музыкальные характеристики изображения, такие как тональность и жанр создаваемой музыкальной композиции~\cite{alg}.
Тональность и жанр --- являются ключевыми параметрами для трансляции изображения в музыку, поскольку они формируют эмоциональную составляющую произведения, и должны быть определены путем анализа цветовой гаммы изображения~\cite{actuality}.

\textbf{Определение тональности произведения}

%todo pic 
Тональность произведения определяется путем нахождения преимущественного цвета изображения и сопоставление его с выбранной схемой соответствия тона и цвета.
В данном алгоритме была выбрана схема А.~Н.~Скрябина~(таблица~\ref{tab:color})~\cite{automatic_sound_generation, colortonote}.

\begin{table}[h]
	\centering
	\caption{Соответствие цвета и тональности по схеме А.~Н.~Скрябина}
	\label{tab:color}

	\begin{tabular}{|l|l|}
		\hline
		\textbf{Цвет} & \textbf{Тональность} \\
		\hline
		Красный & C-dur \\
		\hline
		Оранжево-розовый & G-dur \\
		\hline
		Желтый, яркий & D-dur \\
		\hline
		Зеленый & A-dur \\
		\hline
		Синий, сапфировый & E-dur \\
		\hline
		Синий, мрачный & H-dur \\
		\hline
		Сине-яркий & Fis-dur \\
		\hline
		Фиолетовый & Des-dur \\
		\hline
		Пурпурно-фиолетовый & As-dur \\
		\hline
		Красный & F-dur \\
		\hline
	\end{tabular}
\end{table}




Для определения преимущественного цвета изображения, традиционно выбирают алгоритм кластеризации K-средних~\cite{alg, web}.
Данный алгоритм имеет следующие особенности: высокое качество кластеризации, возможность эффективного распараллеливания, существует множество модификаций, число кластеров надо знать заранее~\cite{clustering, tyurin_cluster_analysis}. 

\textbf{Алгоритм кластеризации K-средних} --- это неконтролируемый метод обучения.
В этом методе данные разбиваются на кластеры на основе их сходства без использования предварительных обозначений~\cite{alg}.


Алгоритм состоит из следующих шагов~\cite{defcluctering}

\begin{enumerate}
	\item Выбирается число \( k \) --- количество кластеров.
	\item Далее случайным образом из заданного изображения выбирается \( k \) точек. На первом шаге эти точки будут считаться "центрами" кластеров. Каждому кластеру соответствует один центр.
	\item Все точки изображения распределяются по кластерам. Вычисляется расстояние от точки до каждого центра кластера (например используя Евклидово расстояние~\ref{eq:evclid}), и точку относят к тому кластеру, расстояние до центра которого будет наименьшим.
	\item Когда все точки изображения распределены по кластерам, происходит пересчет центров кластеров. В качестве нового центра кластера берется среднее арифметическое всех точек, принадлежащих кластеру.
\end{enumerate}

Пункты 3 и 4 повторяются до тех пор, пока не будет выполнено условие в соответствии с некоторым критерием остановки:
\begin{itemize}
	\item кластерные центры стабилизировались, то есть все наблюдения принадлежат кластеру, которому принадлежали до текущей итерации;
	\item число итераций равно максимальному числу итераций.
\end{itemize}

В алгоритме k-средних ставится цель минимизировать полную внутриклассовую дисперсию:

\begin{equation}
	V = \sum_{i=1}^{k} \sum_{X_j \in C_i} (X_j - \mu_i)^2
	\label{eq:disperion}
\end{equation}

где \(X_j\) --- векторы характеристик, \(k\) --- количество кластеров, \(C_i\) --- кластеры, \(\mu_i\) --- центры кластеров.


В качестве метрики для данной задачи традиционно берут евклидово расстояние~\cite{defcluctering, pruf_clustering, feature_representations}.

Евклидово расстояние между точками $x_i$ и $y_i$ в $n$-мерном пространстве выражается следующей формулой:
\begin{equation}
	\rho(x_i, y_i) = \sqrt{\sum_{i=1}^{n} (x_i - y_i)^2}
	\label{eq:evclid}
\end{equation}

Результаты алгоритма кластеризации K-средних~\cite{academic_performance}:

\begin{itemize}
	\item Центроиды кластеров \(k\), которые можно использовать для маркировки новых данных;
	\item Метки для обучающих данных (каждая точка данных назначается одному кластеру).
\end{itemize}



\textbf{Определение жанра произведения}

Жанр произведения определяется так же по преимущественному цвету изображения. 
В статье~\cite{automatic_sound_generation} дано соответствие между преимущественным цветом и жанром произведения~(таблица~\ref{tab:rejected_colors}).

\begin{table}[h]
	\centering
	\caption{соответствие цвета и жанра}
	\label{tab:rejected_colors}

	\begin{tabular}{|c|c|}
		\hline
		\textbf{цвет} & \textbf{Музыкальный жанр} \\
		\hline
		Синий & Блюз \\
		\hline
		Зеленый & Классика \\
		\hline
		Красный & Рок \\
		\hline
		Желтый & Классика \\
		\hline
	\end{tabular}
\end{table}

\textbf{Определение схем соотнесения цвета и ноты}  

В статье~\cite{colortonote} описывается множество подобных схем, например соотнесение цветов и нот по И. Ньютону, он искал связь между солнечным спектром и музыкальной октавой, сопоставляя длины разноцветных участков спектра и частоту колебаний звуков гаммы, таблица~\ref{tab:Newton}. В алгоритме будет использована эта схема, так как традиционно выбирают ее~\cite{alg, actuality, web, big}. 
	
	\begin{table}[ht]
		\centering
		\caption{Соотнесение цветов и нот по И. Ньютону}
		\label{tab:Newton}

		\begin{tabular}{|c|c|}
			\hline
			\textbf{Цвет} & \textbf{Нота} \\ 
			\hline
			Красный & До \\
			\hline
			Фиолетовый & Ре \\
			\hline
			Синий & Ми \\
			\hline
			Голубой & Фа \\
			\hline
			Зеленый & Соль \\
			\hline
			Желтый & Ля \\
			\hline
			Оранжевый & Си \\
			\hline
		\end{tabular}
	\end{table}

\subsection{Алгоритм генерации музыкального фрагмента}

Для генерации музыкального фрагмента можно применить множество различных
математических методов: марковские цепи, генетический алгоритм, порождающие
грамматики и другие. В данной работе для решения этой задачи будет применена рекуррентная архитектура искусственной нейронной сети.
%поскольку они позволяют относительно просто обучать различные модели и создавать композиции на основе скрытых взаимосвязей в музыкальных данных, которые не всегда формализуемы и видны человеку.
Этот тип сетей отлично выявляет и воспроизводит взаимосвязи в неструктурированных данных, что делает его эффективным для генерации музыки~\cite{nikitin2022, tolstov, toktarbekov2022}.

\textbf{Рекуррентные нейронные сети}

Рекуррентные нейронные сети содержат обратные связи. На рисунке~\ref{img:rnn} представлен фрагмент нейронной сети $A$ принимает входное значение $Xt$ и возвращает значение $ht$. 
Наличие обратной связи позволяет передавать информацию от одного шага сети к другому.
Это обеспечивает некоторую форму памяти, что важно для генерации музыки, где каждая нота связана с предыдущими состояниями~\cite{nikitin2022, lindigrin2019}. 
 
\includeimage
	{rnn} % Имя файла без расширения (файл должен быть расположен в директории inc/img/)
	{f} % Обтекание (без обтекания)
	{h} % Положение рисунка (см. figure из пакета float)
	{0.2\textwidth} % Ширина рисунка
	{Фрагмент рекуррентной нейронной сети} % Подпись рисунк


Однако у рекуррентных нейронных сетей возникает проблема долгосрочных зависимостей, которая проявляется в ограниченной способности корректного вычисления текущего состояния на основе далеких прошлых данных~\cite{nikitin2022, tolstov, lindigrin2019}.

\textbf{LSTM}

Для решения данной проблемы были изобретены сети с долгой краткосрочной
памятью или LSTM (от англ. long short-term memory)~\cite{lindigrin2019}. В отличие от обычных рекуррентных нейронных сетей, где повторяющийся модуль представляет собой функцию, в сетях LSTM этот модуль содержит четыре взаимодействующих между собой слоя: входной слой (фильтр), выходной слой, забывающий слой и ячейка памяти. Основной целью этих слоев является предотвращение данных от перезаписи и забывания. Такие нейронные сети наиболее хорошо подходят для анализа сложных структур, например, текстовых и музыкальных данных, а также предсказания временных рядов~\cite{nikitin2022}.
Подробное описание и пошаговый разбор сети LSTM, предоставлены в статье~\cite{actuality}.

\textbf{Обучение сети LSTM}

В алгоритме использован \textit{midi} формат для обучения и предсказания композиций, так как нейронная сеть работает исключительно с числовой информацией. 
Особенности данного формата: данные о начале и окончании звучания конкретной ноты или набора нот представлены в числовом виде, при этом высота и ее название специальным образом зашифровано в числовые значения. Такой подход не требует дополнительных преобразований входных обучающий файлов --- их сразу можно подготовить в \textit{midi} формате~\cite{nikitin2022}.

Сценарий обучения сети описан в статье~\cite{nikitin2022} и выглядит следующим образом: на заранее подготовленном наборе музыкальных композиций в \textit{midi} формате для определенного жанра, 
обучаем сеть и сохраняем модель в отдельный файл для данного жанра.
Обучение для определенного жанра позволит получить более подходящий выход сети.
На рисунке~\ref{img:alg} представлена схема верхнеуровневого алгоритма для генерации фрагмента музыкального произведения по изображению. Учтено, что доступны обученные модели на различных наборах данных по жанрам~\cite{nikitin2022}.

\includeimage
	{alg} % Имя файла без расширения (файл должен быть расположен в директории inc/img/)
	{f} % Обтекание (без обтекания)
	{H} % Положение рисунка (см. figure из пакета float)
	{0.35\textwidth} % Ширина рисунка
	{Верхнеуровневый алгоритм для генерации фрагмента музыкального произведения по изображению} % Подпись рисунк





%Цель работы достигнута: описан метод генерации фрагмента музыкального произведения по изображению.
\subsection*{Вывод}
Предложенный метод находит приближенную функцию многозначного отображения, которая описана формулой~(\ref{eq:test}).

Метод генерации фрагмента музыкального произведения по изображению включает два основных этапа: анализ изображения и генерация музыкального фрагмента. 
На этапе анализа изображения получаются следующие характеристики: тональность, жанр и последовательность первых нот. Эти данные необходимы для последующей генерации музыкального фрагмента.
Для определения тональности используются алгоритм кластеризации K-средних и соответствие цвета и тональности с учетом схемы А.~Н.~Скрябина.
Определение жанра также базируется на алгоритме кластеризации K-средних и соответствии цвета и жанра.
Генерация музыкального фрагмента выполняется с использованием рекуррентной архитектуры нейронной сети LSTM. 
Обучение сети осуществляется на подготовленном наборе музыкальных композиций в \textit{midi} формате для конкретного жанра. 


\if 0

\\
В предыдущем разделе были получены характеристики композиции, такие как тональность, жанр и последовательность первых нот.
Тогда генерация композиции музыкального фрагмента с использованием нейронной сети LSTM по заданным характеристикам состоит из следующих шагов~\cite{nikitin2022}.

\begin{enumerate}
	\item По заданному жанру, тональности и последовательность первых нот определяем нужную модель для использования.
	\item загружаем модель в память.
	\item Генерируем композицию в \textit{midi} формате с использованием загруженной модели.
\end{enumerate}


Тональность произведения определяется двумя преимущественными цветовыми характеристиками --- оттенок и цветовая группа, а темп --- яркость и насыщенность~\cite{web}~(таблица~\ref{tab:color_music})

\begin{table}
	\centering
	\caption{Соотношение цветовых и музыкальных характеристик}
	\label{tab:color_music}
	\small 
	\begin{tabular}{|c|c|}
		\hline
		\textbf{Цветовые характеристики} & \textbf{Музыкальные характеристики} \\
		\hline
		Оттенок (красный, синий, жёлтый…) & Нота (до, до-диез, ре, ре-диез ...) \\
		\hline
		Цветовая группа (тёплый/холодный) & Музыкальный лад (мажор/минор) \\
		\hline
		Яркость & Октава ноты \\
		\hline
		Насыщенность & Длительность ноты \\
		\hline
	\end{tabular}
\end{table}

 согласно выбранной схеме соотнесения цветов и нот, а также результатах, полученных на предыдущих шагах, определяем тональность произведения.
 
 %todo
 
\fi