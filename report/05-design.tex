\chapter{Классификация существующих решений}

Для краткости записи в данной таблице используются следующие обозначения описанных критериев:
\begin{itemize}
	\item К1 --- необходимость обучения перед генерацией фрагмента; 
	\item К2 --- возможность генерации в режиме реального времени;
	\item К3 --- входные данные. 
\end{itemize}


\begin{table}[h]
	\centering
	\caption{Сравнение различных методов}
	\begin{tabular}{|p{5cm}|c|c|p{7cm}|}
		\hline
		\textbf{Метод} & \textbf{К1} & \textbf{К2} & \textbf{K3} \\ \hline
		Алгоритм Бъорклунда & Нет & Нет & Длина музыкального фрагмента, количество сигналов, аккорд \\ \hline
		Генерация с помощью клеточного автомата & Нет & Нет & Набор правил перехода состояний, набор аккордов, соответствующий правилам \\ \hline
		Метод случайного блуждания & Нет & Нет & Длина музыкального фрагмента, множество аккордов \\ \hline
		Метод Марковских цепей & Да & Да & Длина музыкального фрагмента, множество аккордов \\ \hline
		Порождающая грамматика & Нет & Нет & Набор правил перехода состояний, набор аккордов, соответствующий правилам \\ \hline
		L-системы 	& Нет & Нет & Набор правил перехода состояний, соответствующий правилам, набор, состоящий из алфавита, аксиомы \\
		\hline
		LSTM & Да & Нет & Тональность, жанр и последовательность первых нот\\ \hline
	\end{tabular}
	\label{table:methods}
\end{table}