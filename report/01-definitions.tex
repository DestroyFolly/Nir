\begin{definitions}
	% мое 
	\definition{Музыкальный тон}{это устойчивый периодический звук. Музыкальный тон характеризуется его длительностью, высотой, интенсивностью (или громкостью) и тембром (или качество)~\cite{umk}.}
	
	\definition{Темп}{это скорость движения в музыке, мера времени в музыке~\cite{umk}.}
	
	\definition{HSB (от англ. Hue, Saturation, Brightness --- тон, насыщенность, яркость)}{цветовая модель, в которой координатами цвета являются: цветовой тон (например, красный, зелёный или сине-голубой), насыщенность (варьируется в пределах 0—100), яркость (варьируется в пределах 0—100)~\cite{rgb}.}
	
	\definition{RGB (от англ. Red, Green, Blue --- красный, зелёный, синий)}{цветовая модель, описывающая способ кодирования цвета для цветовоспроизведения с помощью трёх цветов, которые принято называть основными~\cite{rgb}.}
	
	\definition{Кластеризация}{это разбиение элементов некоторого множества на группы по принципу схожести. Эти группы принято называть кластерами~\cite{defcluctering}.}
	
	\definition{Объект}{элементарная группа данных, с которой оперируют алгоритмы кластеризации. Каждый объект описывается вектором характеристик~\cite{defcluctering}:
		$X = \{x_1, x_2, \ldots, x_m\}$, где $m$ --- количество характеристик объекта. Компоненты $x_i$, где ($i = 1, 2, \ldots, m$) являются отдельными характеристиками объекта, как правило, представленными количественными признаками (например, координаты для точки, цветовые компоненты для цвета и т. д.)~\cite{defcluctering}.
	}
	% СТепа
	\definition{Искусственный интеллект (ИИ)}{это технология, которая имитирует человеческое поведение, чтобы выполнять задачи и постепенно обучаться, используя собранную информацию~\cite{AI}.}
	
	\definition{Компьютерная музыка}{музыка, созданная с использованием электромузыкальных инструментов и электронных технологий последних десятилетий XX века~\cite{music}.}
	
	\definition {Музыкальная нота}{это моментальный снимок периодической звуковой волны определенной частоты~\cite{notes}.}

	\definition{Аккорд}{это несколько различных по высоте звуков (более двух) воспроизводимых одновременно~\cite{chords}.}	
\end{definitions}