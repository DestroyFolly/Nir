\chapter*{ВВЕДЕНИЕ}
\addcontentsline{toc}{chapter}{ВВЕДЕНИЕ}

Музыка --- это удивительное искусство, которое оказывает значительное влияние на наши эмоции и настроение.
Она имеет способность переносить нас в другие миры и вызывать самые разнообразные чувства. Изобретение фонографа, радио, телевидения и интернета сделало музыку всеобщей и непреходящей. Звуки и мелодии распространяются между континентами взаимодействуя и обогащая друг друга. Однако, за этой привлекательной формой может скрываться глубокий и сложный процесс написания музыки.

Задача генерации музыкального фрагмента стала актуальной, начиная со второй половины 20 века и является такой и по сей день. Крупные корпорации, такие как Google и Sony разработали нейросети MusicLM и FlowMachine, с помощью которых пользователи могут создавать музыкальные композиции.

Цель данной работы --- провести анализ методов генерации музыкального фрагмента.

Для достижения поставленной цели необходимо решить следующие задачи:
\begin{itemize}
	\item провести анализ предметной области;
	\item провести обзор существующих методов и алгоритмов генерации музыкального фрагмента;
	\item сформулировать критерии сравнения методов генерации музыкального фрагмента;
	\item классифицировать описанные методы.
\end{itemize}